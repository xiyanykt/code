\documentclass{article}
\usepackage{lipsum}
\usepackage{ctex}
\usepackage{geometry}
\thispagestyle{empty} % 当前页不显示页码
\geometry{a4paper,scale=0.7}
\begin{document}
\newpage
\section*{\textsf{问题B. }\textrm{DDE}}
\begin{tabular}{ll}
	\fontsize{10pt}{10pt}\texttt{输入文件:} & \fontsize{10pt}{10pt}\texttt{标准输入}          \\
	\fontsize{10pt}{10pt}\texttt{输出文件:} & \fontsize{10pt}{10pt}\texttt{标准输出}          \\
	\fontsize{10pt}{10pt}\texttt{时间限制:} & \fontsize{10pt}{10pt}\texttt{2000 ms}       \\
	\fontsize{10pt}{10pt}\texttt{内存限制:} & \fontsize{10pt}{10pt}\texttt{256 megabytes} \\
\end{tabular}
\subsection*{\textsf{问题描述}}
判断是否可以通过执行一定次数(可能为零次)操作,使数组中的所有元素相同。经过这些操作后,数组中的所有元素将相等。

给定一个由$n$个正整数组成的数组$a$。你可以对它执行以下操作:

1. 选择一对元素$a_i$和$a_j$($1 \le i, j \le n$且$i \neq j$);

2. 选择整数$a_i$的一个除数,即一个整数$x$,使得$a_i \bmod x = 0$;

3. 将$a_i$替换为$\frac{a_i}{x}$,并将$a_j$替换为$a_j \cdot x$。

判断是否可以通过执行一定次数(可能为零次)操作,使数组中的所有元素相同。

例如,考虑数组$a = [100, 2, 50, 10, 1]$,包含$5$个元素。对它执行两次操作:

1. 选择$a_3 = 50$和$a_2 = 2$,$x = 5$。将$a_3$替换为$\frac{a_3}{x} = \frac{50}{5} = 10$,并将$a_2$替换为$a_2 \cdot x = 2 \cdot 5 = 10$。结果数组为$a = [100, 10, 10, 10, 1]$;

2. 选择$a_1 = 100$和$a_5 = 1$,$x = 10$。将$a_1$替换为$\frac{a_1}{x} = \frac{100}{10} = 10$,并将$a_5$替换为$a_5 \cdot x = 1 \cdot 10 = 10$。结果数组为$a = [10, 10, 10, 10, 10]$。

执行这些操作后,数组$a$中的所有元素都变成了$10$。
\subsection*{\textsf{输入}}
输入的第一行包含一个整数$t$($1 \le t \le 2000$)——测试用例的数量。

接下来是每个测试用例的描述。

每个测试用例的第一行包含一个整数$n$($1 \le n \le 10^4$)——数组$a$中的元素数量。

每个测试用例的第二行包含恰好$n$个整数$a_i$($1 \le a_i \le 10^6$)——数组$a$的元素。

保证所有测试用例中$n$的总和不超过$10^4$。
\subsection*{\textsf{输出}}
对于每个测试用例,输出一行:

如果可以通过执行一定次数(可能为零次)操作使数组中的所有元素相等,则输出"YES";
否则,输出"NO"。

\noindent
\subsection*{\textsf{样例}}
\noindent
\begin{tabular}{| p{7cm} | p{7cm} |}
	\hline
	{\quad\quad\quad\quad\quad\quad\quad\  标准输入} & {\quad\quad\quad\quad\quad\quad\quad\  标准输出} \\
	\hline
	7

	5

	100 2 50 10 1

	3

	1 1 1

	4

	8 2 4 2

	4

	30 50 27 20

	2

	75 40

	2

	4 4

	3

	2 3 1                                        & YES

    YES
	
    NO
	
    YES
	
    NO
	
    YES
	
    NO                                                                                          \\
	\hline
\end{tabular}

\end{document}
% pdflatex statement.tex